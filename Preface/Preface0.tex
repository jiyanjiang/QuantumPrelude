\section{写在前面的话}

首先这不是一门课程,但如果我试图给非物理专业的普通人讲清楚量子力学,又必然碰到这样一个困难。物理专业的学生是经过将近三年的艰苦准备才正式开始这样一个主题的学习的,而要完全掌握又至少需要经过将近100学时的授课和至少双倍时间的自学和练习。我们凭什么能很轻松地对一个非专业的普通人讲清楚量子力学呢。

也有人说能否对一个外行讲明白,是“是否懂得”某门学问的真正标志,我正在做这样一个努力,这个努力看起来就是,我花了几乎90\%的时间在说思维,在说概念和数学,而几乎不涉及大家想象中的量子力学,但这真的很重要,你必须用你能Handle,你能想象的经验逐渐进入思维的状态,而一旦你养成了思维的习惯,并真正喜欢它,那你就会轻松地进入任何一个你感兴趣的领域(包括量子力学),并欣赏它,而是否能深入并进而有所造诣,则要看个人的天赋、勤奋和是否准备在这个领域投入“十万小时”,这里并无太多捷径,每个人都会有属于自己的十年创造力的高峰,至于是否会创造出属于自己的奇迹,也许并不重要。

我仍然认为一个“外行”是可以懂量子力学的,前提是如果你对我这样的一份讲义感兴趣。

首先让我们由一个假想的课堂开始,看看我们到底缺什么?或这样一门课程在整个物理学中的地位如何?先对此有个大致了解,然后再展开我们的讨论。(如果你对某节不感兴趣,大可跳过不看,进入下一节,我尽量使每节具有独立阅读的可能性。)

\subsection{对课程的简介}

量子力学是大学物理系的基础课程,是一门承前启后的课程。在物理系的教学计划里它一般放到普通物理之后,在理论物理四大力学中是最后一门学习的,但会放在固体物理、量子场论等更高级的课程之前。

正常情况下,我们假设一个学生在开始学习量子力学之前已经修过以下课程:

\begin{itemize}

\item 普通物理:力学,热学,光学,电磁学,原子物理学

\item 理论物理:分析力学,电动力学,热力学与统计物理 
 
\item 数学课程:微积分,微分方程,线性代数,数学物理方法 
 
\item 实验课程:普通物理实验,近代物理实验,模拟和数字电路实验

\end{itemize}

很大程度上理论物理是对普通物理的重复,理论物理更重视建立数学的形式体系,而普通物理则更注重描述现象并阐述概念的由来。

力学和分析力学是研究运动的,典型研究对象是质点、刚体和机械波。

热学和热力学与统计物理都是研究热现象的,典型研究对象是理想气体。

电磁学和电动力学都是研究电磁现象的,但我们必须知道光现象是一种特殊的电磁现象,所谓光波就是我们肉眼可见的电磁波,对应波长范围是770纳米-350纳米。

以上三个领域在量子力学出现之前就已经成熟了,构成了我们今天所说的经典物理。

\begin{itemize}

\item 经典力学:研究运动和力,典型成就是万有引力定律,我们利用万有引力定律可以解释行星的运动,计算卫星的轨道。

\item 经典电磁学:研究电现象、磁现象和光现象,典型成就是麦克斯韦方程组,广播和通讯等都是基于电磁学的技术应用。

\item 经典统计:研究热现象,典型成就是热力学三大定律和分子运动论。经典统计典型运用的领域是化学。

\end{itemize}

三个领域表面看各不搭界,各有各的基本理论,质点的运动,不同于电磁场的运动,也不同于“熵的增加”。

仔细想想,我们对自然界有三套基本法则,会感到不舒服,最好它们有共同的基础,这才让我们心安。

再者,对机械运动、电磁运动和热运动的划分,仅仅是我们人类的主观界定,很大程度上是为了简化问题,而在现实世界中我们终将会遭遇某个问题它同时会牵涉到不止一种运动,此时我们至少应该期待现象能够被这三套法则以互相不矛盾的方式得到解释。

或者干脆说,我们将会很期待这类牵涉到不止一类基本法则的典型问题的出现。实际上有很多这类问题,最有名的当属“黑体辐射”和“寻找以太”。

黑体辐射研究的对象是处于特定温度下的电磁辐射场。这就涉及到经典电动力学和经典统计两个领域。

以太是假想出的可以承载光波(或电磁波)的介质,就像声波需要空气、水波需要水一样,这就涉及了经典力学和经典电动力学两个领域。

这两个问题最有名,100多年前,物理学家没有找到以太,也没能用经典统计+经典电动力学解释黑体辐射,这就是开尔文勋爵(即威廉·汤姆逊,1824-1907)在上世纪初所说的漂浮在经典物理学上空的两片乌云。

几乎所有的物理教科书上都会这么说,因为这两片乌云与量子力学和相对论直接相关。今天物理学的基础正是量子力学和相对论,就像100多年前我们说物理学的基础是经典力学、经典电动力学和经典统计一样。比如:今天我们要求一个物理理论必须符合相对论,并且可以量子化。

但对引力,我们还没有找到一个令人满意的可以量子化的理论。

寻找以太的失败让爱因斯坦的狭义相对论脱颖而出,在这个框架里以太是个多余的概念,这个词被取消了,自然也不需要找了。狭义(special)就是特殊的意思,即我们只研究匀速运动。广义相对论研究加速运动,而加速运动和处在引力场中又是无法区分的,爱因斯坦就是基于这类讨论建立起了一个几何化的引力理论。

相对论本身不包含量子化的概念,从这个角度说相对论是个经典的理论。并且相对论基本上是爱因斯坦一个人的功劳,虽然说洛伦兹、彭加勒也曾接近过狭义相对论,但接近并非得到。

\subsubsection{量子}

普朗克(Max Planck,1858-1947)为了解释黑体辐射现象第一个使用了“量子”这一概念,他假想处在某个温度下的电磁辐射场与其他物质,比如包裹着这个电磁辐射场的外壁会发生能量的交换,这个能量必须是一份一份的。

一份就是一个量子(quanta)。

这就好比我们去酒馆买啤酒喝,我们只能一杯一杯地买。一杯就是一品脱(1pint,大概相当于500毫升)。

我们需要记住电磁辐射就是光,光就是电磁辐射,为了表述的方便,我们将一会儿说光,一会儿说电磁辐射,完全随意。这里的问题是光是什么?光是本身就是一份一份的呢?还是光仅仅在和物质发生相互作用的时候才以一份、一份这个方式进行交换的。

普朗克强调的是后者,这又让人很不舒服,因为这太人为了,光凭什么如此行为需要解释。爱因斯坦的观点和普朗克不同,他认为光本身就是以一份一份的形式存在的。

这就好比我们抓起一把沙子,沙子本身很细小,一把能抓起成千上万个,但这不否认光本身有个计量的最小单元,这个单元我们权且称其为光子(photon),一个光子就是一个量子,我们一般用能量的多少来描述量子,一个量子就是一个光子携带的能量,它正比于光的频率。

我们用大写英文字母$E$来表示能量,小写希腊字母$\nu$表示频率,而比例因子被记做$h$,为了荣耀普朗克,我们称之为普朗克常数。

现在一个光子(或一个光量子)携带的能量就是:

\begin{equation}
E=h \nu
\end{equation}

在国际单位制中能量的单位是焦耳,能量是物理世界中的“货币”,它可以有多种形态存在,并可以互相转换,但在转换过程中没有“花账”和“黑洞”,我们需要做的是把转换因子定义好,这样就可得到能量守恒的概念了。

对给定物理系统,能量可以互相转换,但总能量必须守恒,如果不守恒我们就得找原因,比如是否有外力对系统做功,比如是否有来自环境的能量流等等。

频率的单位是秒分之一,记做$s^{-1}$,也叫赫兹(Hz,为了纪念用实验证明了电磁波就是光波的德国物理学家赫兹,1857-1894),1Hz指的是在一秒钟内振动重复了1个周期,10Hz指的是1秒内振动重复了10个周期,那么周期$T$和频率$\nu$的关系就是:

\begin{equation}
T = \frac{1}{\nu}
\end{equation}

普朗克常数的单位是“焦耳·秒”,记做$J \cdot s$,普朗克常数的大小是:

\begin{equation}
h = 6.626 \times 10^{-34} J \cdot s
\end{equation}

普朗克常数是个很小的数(“1后面34个零”分之一),这解释了为什么我们平时并不觉得光是一份一份的。比如一个以10米每秒速度奔跑的人具有的动能大致相当于$10^22$个可见光光子。

爱因斯坦(Albert Einstein,1879-1955)认为光本身就是以量子形态存在的,光子是像粒子一样的实体,它能集中地把能量给电子,只要光子的能量足够强,换句话说就是频率足够高(波长足够短),光就能把电子从金属里一脚踢出。这就是爱因斯坦对光电效应的解释,非常简单、直观。

这里我们发现一个问题,量子力学的研究和物质有关,它涉及我们对物质结构的理解,它寻求对物质性质的解释等等。

经典物理不是典型的关于物质的理论,比如经典力学研究运动,它把物质抽象为具有质量的点,或这样点的集合,对经典力学来说,一颗铜行星还是一颗铁行星关系不大,我们都是用诸如位置、速度和质量这类物理量来描述。

化学是研究物质的,化学家会关心物质的颜色、是否有金属光泽,导电性如何等等。但仅仅是这些只能算是资料的罗列,还算不得真正的科学。热力学和统计物理为化学提供了部分理论基础,但远远不够,特别的,热力学和统计物理不涉及化学里的核心问题,即为什么一种元素会和另一种元素那么的不一样?或我们为什么会有元素周期律,这种伴随着元素越来越重,元素的化学性质周期性的由活泼到不活泼,再重新开始由活泼到不活泼……的周期性。

这些问题很大程度是由原子物理回答的,原子物理是一种关于物质的理论,假如我们对原子施加的能量不够大,原子作为物质存在的一个基本单元会保持稳定,原子的性质主要由原子核外的电子决定,而电子的运动应该由什么样的物理法则来描述呢?

最简单的原子是氢原子,原子核外只有1个电子,原子核带正电,电子带负电,它们之间会有吸引力,表面看来这和求解月亮如何围绕地球运动一样,区别仅仅是电磁相互作用替代了万有引力。

根据经典电动力学,如果电子围绕原子核做圆周运动的话,它将会向外辐射能量,以电磁辐射的形式(或说以光的形式),因为电磁相互作用很强,电子本身具有的动能将会以很快的速度被辐射掉,电子因为动能耗尽,会像一颗运动速度过慢的卫星那样一头掉到原子核上。这句话翻译过来就是原子不存在了。

实际上原子相当稳定,比如我们的身体里就有很多很多氢原子,它们显然是非常稳定的,并且这构成了我们能安然坐在这里思考物理问题的前提。

看来经典理论不足以解释原子。

\subsubsection{量子力学是青年文化的产物}

量子力学与相对论以及从前任何一个物理理论的不同在于它没有一个明确的创建人,我们没法把量子力学和某一位物理学家的名字联系起来,就像我们把经典力学和牛顿,经典电动力学和麦克斯韦,相对论和爱因斯坦联系起来一样。

为了说清楚量子力学的创建史,我们必须提及至少几十个物理学家,他们都曾做出不可以忽略的重要贡献,可以说量子力学的创建为我们贡献了最多的物理学大师。

从普朗克和爱因斯坦开始,玻尔、索末菲、玻恩、德布罗意、海森堡、薛定谔、狄拉克、泡利、约丹、费米、玻色……这还仅仅是理论家。

还有很多实验家的名字同样不能忽略:J J 汤姆逊、卢瑟福、密立根、康普顿、劳厄、布拉格父子、布拉开特、劳伦斯……

如果我们总结的话,这主要是一群生活在西北欧的白种青年男性,大多在20-30岁之间,他们之间紧密合作、互通信息,同时又相互竞争。

量子力学是这种青年文化的产物,先是卢瑟福和他的孩子们通过散射实验确立了原子的有核模型,继而是卢瑟福的孩子玻尔猜出了一个能够解释氢原子光谱主要特征的玻尔模型。

海森堡是玻尔、玻恩和索末菲共同的孩子,他从概念上摒弃了那些很难被观测的物理量,如原子的位置,转而从原子的光谱现象出发构建理论,这是一种反玄学的态度,即让物理公式中只出现那些可被测量的物理量。海森堡第一个取得突破,他找到了一个可以表述为矩阵的数学关系来计算原子的光谱现象。

几乎与此同时,薛定谔由德布罗意的物质波概念出发,把电子的运动想象为一种可以用波描述的对象。然后建立了一个对波函数适用的偏微分方程——薛定谔方程。

对物理学家来说,波动是一种很熟悉的图像。对普通人来说也是如此,琴弦上的振动是波,两列水波相遇是波,声波是波,电磁波也是波等等。

这里我们把电子的运动状态用一个波函数来描述,这和说我们把电子想象成一个波还是有区别的。物理学家的意思是电子本身还是粒子,只是它的运动需要借助波的语言。在此基础上我们需要发展成套的数学工具来描述电子的性质,比如:如何由一个波函数知道电子的能量等等。

我们把海森堡的方案叫矩阵力学,把薛定谔的方案叫波动力学,这都是很直接的命名,如果你解矩阵方程的话,就是矩阵力学,如果你解波动方程的话就是波动力学。同时我们也理解为什么一个物理系的学生在学量子力学之前必须先学线性代数和数理方法。线性代数是研究矩阵运算的,而数理方法教我们如何求解波动方程。

量子力学的形式体系是由狄拉克最终完成的,狄拉克比海森堡稍小,他的教育背景是工程学和应用数学。他发明了很多方便的记号去表达量子力学,同时他重新建立起了量子力学与经典力学的联系。

在经典力学中我们有多种数学形式来描述一个物理系统,比如我们可以选择哈密顿的形式,选择拉格朗日的形式等等。狄拉克证明了只要稍加改动——引入普朗克常数$h$——就能把经典力学量子化。

比如我们让位置和动量满足如下关系:

\begin{equation}
xp - px = i \frac{h}{2 \pi}
\end{equation}

这里$i$表示纯虚数,$x$和$p$分别表示位置和动量。

Again,因为普朗克常数太小了,我们以前都没有注意到原来自然是如此行事的。在此意义下,狄拉克推广了经典力学的数学语言使之重新适用于电子的世界。

狄拉克的体系是完结性的,以后量子力学还有发展,但我们使用的语言和思维方式是狄拉克规定的。比如我们迄今为止还没有考虑相对论,对电子考虑相对论后,我们将得到一个相对论性的方程。这也是狄拉克的工作。

作为一个课程,我们将预先设定我们的任务是讨论非相对论性的量子力学,我们也不讨论对场的量子化,不讨论量子统计,不讨论量子力学利用于金属或其他固态结构。这分别对应于量子场论、量子统计、固体理论等其他后续课程。

量子力学建立后,物理学的进展主要集中在三个领域:

\begin{itemize}

\item 场论与粒子物理:典型成就有量子电动力学,标准模型,量子色动力学等。

\item 宇宙学和广义相对论:如何把引力量子化?

\item 凝聚态物理:这个主要是应用推动的,近二十年来的信息社会就是建基于凝聚态物理学的进展。

\end{itemize}

如果大家想了解最新的物理新闻的话,可以访问英国物理学会的PhysicsWorld(\url{http://physicsworld.com/})。

最后,我要强调的是我们必须做练习,学习量子力学必须要拿起笔来做计算,这些计算其实都很简单,但如果你不算,不知道过程,你就不能说你已经懂了。


\subsection{科学计数法}

在物理中我们经常会碰到很大的数或很小的数,比如整个宇宙中质子的数目就是个很大的数,这个数字是大约$10^{80}$,我们不管这个数字是怎么来的,先来理解$10^{80}$是什么意思。

$10^1$就是10,即1后面有1个零,$10^{80}$就是1后面有80个零,如果老老实实全写出来的话,得写四行,而$10^{80}$就很简短。它暗含的意思是甭管多大,它反正是个有限的数,我们很容易相像比它再大是个什么概念,比如2倍就是再乘以个2。

在物理中常见的比较大的数还有阿佛加德罗常数,一般记做$N_A$,它表示的是常温常压下22.4升氢气里面氢气分子的个数。

\begin{equation}
N_A = 6.02 \times 10^{23}
\end{equation}

这个数大概是6后面再跟23个0,也是个很大的数。我们也管$N_A$个氢气分子叫做1摩尔氢气分子,类似地$N_A$个电子就是1摩尔电子,这个概念在化学里面经常用。

1升是容积单位,相当于是1立方分米,就是长宽高各1分米(即0.1米)所包围的体积。22.4升相当于多少立方米呢?

\begin{equation*}
22.4 \times (0.1)^3 = 22.4 \times 0.001 = 0.0224 (m^3)
\end{equation*}

22.4升相当于0.0224立方米,我们也常常把它改写为:

\begin{equation*}
2.24 \times 10^{-2} m^3
\end{equation*}

这个也是科学计数法,$10^{-1} = 0.1$,表示10分之1,1的前面有1个0,$10^{-2} = 0.01$,表示百分之一,即1的前面有两个0,……$10^{-100}$,表示$10^{100}$分之一,即1的前面有100个0。

现在我们来估算在常温常压下,每个氢气分子平均而言占多少体积:

\begin{equation*}
\frac{2.24 \times 10^{-2}}{6.02 \times 10^{23}} = 0.372 \times 10^{-25} = 3.72 \times 10^{-26} (m^3)
\end{equation*}

这是个很小的数,$10^{-26}$表示$10^{26}$分之一,即1的前面有26个0。

如果把$3.72 \times 10^{-26} (m^3)$看作是个小立方体,它的边长就是体积的开立方:

\begin{equation*}
( 3.72 \times 10^{-26} )^{\frac{1}{3}} = (37.2 \times 10^{-27})^{\frac{1}{3}}
\end{equation*}

立方根的英文是cube root,我们只要在谷歌浏览器里键入:

\begin{verbatim}
    cube root 37.2
\end{verbatim}


就可得到结果:3.338,因此:

\begin{equation*}
( 3.72 \times 10^{-26} )^{\frac{1}{3}} = (37.2 \times 10^{-27})^{\frac{1}{3}} = 3.338 \times 10^{-9} (m)
\end{equation*}

$10^{-9} m$表示1米的$10^9$分之一,$10^{-9} m$也叫1纳米。这个计算说明,常温常压下每个气体分子大致占据边长为3.3纳米见方的空间。换句话说分子和分子之间的平均间距也是大约3.3纳米。

\subsection{长度}

3.3纳米当然很小,但这毕竟是物质以气态存在时分子间的平均间距。我们还可以估算物质以固态存在时原子间的平均间距,这个距离更小,小到约$10^{-10} m$,即1米的$10^{10}$分之一,我们管$10^{-10} m$叫1埃。

\begin{table}[htdp]
\caption{常见长度单位}
\begin{center}
\begin{tabular}{|c|c|}
\hline
1公里 或 1千米($km$) & $10^3$ 米 \\
1米($m$) & $10^0$ 米 \\
1分米($dm$) & $10^{-1}$ 米 \\
1厘米($cm$) & $10^{-2}$ 米 \\
1毫米($mm$) & $10^{-3}$ 米 \\
1微米($\mu m$) & $10^{-6}$ 米 \\
1纳米($nm$) & $10^{-9}$ 米 \\
1埃($\overset{\circ}{A}$) & $10^{-10}$ 米 \\
1飞米($fm$) & $10^{-15}$ 米 \\
\hline
\end{tabular}
\end{center}
\label{default}
\end{table}%

关于长度我们还可作如下讨论:

\begin{itemize}
\item 1米是长度的主单位,它和我们人自身的尺度相当。

\item $0.1$毫米是人肉眼可清晰分辨的极限,这首先是个事实,即我们在30厘米附近看东西,能够分辨的两个物体的最小间距是大约$0.1$毫米,再小两个东西就要融为一体了。其次这个$0.1$毫米是可以用光学知识估算出来的。

光学中的角分辨本领公式:

\begin{equation}
\theta = 1.22 \frac{\lambda}{D}
\end{equation}

$\theta$是角分辨本领,即可以分辨的最小张角,$\lambda$是光波波长,因为眼睛是通过接受可见光信号来感知事物的,这里我们可以取光波波长为$550 nm$,$D$是瞳孔大小,只有射到瞳孔上的光才是有效的,我们取$D$为2mm。

\begin{equation*}
1.22 \times \frac{550 \times 10^{-9} }{2 \times 10^{-3}} = 3.36 \times 10^{-4}
\end{equation*}

人眼的明视距离是大约30cm,即我们在30cm左右看近处的东西最舒服。$\theta$是张角,我们用$\theta$再乘以明视距离就可获得人肉眼的分辨本领了。

\begin{equation*}
3.36 \times 10^{-4} \times 0.3 \approx 1 \times 10^{-4} (m)
\end{equation*}

即人肉眼分辨的极限是0.1mm,再小的东西我们就必须借助显微镜了。

\item 光波波长的范围是350-770纳米。即略小于1微米($10^{-6}$米)。

\item 芯片制造技艺在最近20年进步很快,从0.5微米、0.35微米、0.25微米、0.18微米、0.15微米、0.13微米、90纳米一直发展到目前最新的65纳米,45纳米和30纳米。即大致相当于几百个原子一个一个挨着的距离。

\item 1埃,或$10^{-10}$米是原子的尺寸,比如氢原子处于基态时的半径就是大约1埃。

\item 1飞米,或$10^{-15}$米是原子核大小的尺度,从$10^{-10}$一下减到$10^{-15}$,原子很空。就像是校园里飞的一只小苍蝇。

\end{itemize}

这里对长度的罗列都是比较短的,还有大尺度的,比如地球的尺寸,地球到太阳的距离(1个天文单位,记作1AU),银河系的尺寸,乃至整个宇宙的尺寸等等。

\begin{table}[htdp]
\caption{典型的大尺度}
\begin{center}
\begin{tabular}{|c|c|}
\hline
珠穆朗玛峰 & 8848m   \\
地球半径 & 6400公里 \\
地球到月球距离 & $3.84 \times 10^5$公里 \\ 
1天文单位 & $1.5 \times 10^8$ 公里 \\
太阳到最近恒星的距离 & 4.22光年 \\
银河系直径 & $1.0 \times 10^5$光年 (10万光年) \\
可见宇宙半径 & 138亿光年 \\
\hline
\end{tabular}
\end{center}
\label{default}
\end{table}%

光年是长度单位,指光传播一年的距离。这里光速是光在真空中传播的速度,光速是通讯(信息有效交换)的上限,但并非是所有速度的上限,比如宇宙大爆炸的瞬间,宇宙本身扩展的速度就远远大于光速。

无论如何,光速$c$是个很大的数字:

\begin{equation}
c = 3.0 \times 10^8 m/s
\end{equation}

即只需要1秒,光就能跑出去$10^8$米,或$10^5$公里,即10万公里。这意味着,在地球上的两个人,无论多远,打电话基本感觉不到时间的延迟。人对时间的延迟大约是$0.1$秒,足够光跑1万公里了,比地球的半径还大。

但如果有两个人分别在地球和月球上,就需要将近4秒才能把信号从地球传导月球上。如果说这个时间延迟还能让人接受的话,这个数字很快将大到无法让人接受。

比如地球和土星的距离是大约10AU,即:$1.5 \times 10^9$公里,或$1.5 \times 10^{12}$米。地球到土星通讯的时间延迟将是:

\begin{equation*}
\frac{1.5 \times 10^{12}}{3.0 \times 10^8 } = 5000(s)
\end{equation*}

5000秒,即大约83分,或1小时23分。可以想象如果未来在土星上会有地球人的殖民地的话,假如在那里发生了叛乱,叛乱的消息传回地球母星将至少需要1小时23分。这意味着我们很难建立一个“太阳帝国”,如果建立了,也很难建立起有效的统治,这是物理法则给出的Polis的上限。

人类布满太阳系应该是未来100-200年内肯定可以实现的,甚至我们还将启航前往最近的恒星,以接近光速的速度航行大约需要5年的时间,届时人类将重新进入一个割据和争霸的时代,同时也是个以“星”为标示的真正多元文化的时代。可以设想那时一个少年的梦想可能会是穷其一生,漫游各星,最后来到地球,一个被文明遗弃的星。


\subsection*{练习}

\begin{enumerate}
\item 

IPhone视网膜屏的分辨率是128PPCM,即128个像素每厘米,MacBook Air屏的分辨率是53PPCM,即53个像素每厘米。

从视觉的角度微米感觉不出IPhone屏上的颗粒,但确实能看见MacBook Air屏上的颗粒,通过计算来解释这个现象。

\item 

光速可表示为:

\begin{equation}
c = \frac{\lambda}{T}=\lambda \nu
\end{equation}

这里$\lambda$是光波的波长,利用这个公式和$E = h \nu$可以估算可见光的能量范围。(已知可见光波长范围是350-770 nm)


\end{enumerate}






%这导致我们必须用科学计数法,科学计数法可以用来表示很大的数,比如1 Googl