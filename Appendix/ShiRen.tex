\section{诗人有一种滔滔不绝的本领}

诗人有一种滔滔不绝的本领。

即便是写物理,讲物理,我们也需要有这种滔滔不绝。

Sidney Coleman就是这样,他滔滔不绝地讲量子场论。

今天我们可通过互联网观看Coleman于1975-1976年讲授的量子场论课程。

\url{https://www.physics.harvard.edu/events/videos/Phys253}

第一次看这段视频的时候,我完全被Coleman吸引住了,不是被他的物理,而是被他的气场和派头。

讲台就是Coleman的舞台,我无法想象竟然有人可以这样讲授,而且是讲物理,这种讲授/表演正是人乐观、自信精神气质的体现,这种精神气质与人运用理性的能力直接相关,而物理正是这种能力、这种精神气质最典型的创造。

Coleman在学生欢笑的气氛中不停地拢他的长头发,动作很大地别上麦克风,夸张地脱去毛衣,点燃一只香烟,深深地呼吸,吐出一个烟圈,在讲台上踱来踱去,不时露出狡黠的笑容和咯咯的笑声。

\begin{figure}[htbp]
\begin{center}
\includegraphics[width=10cm]{Appendix/Scoleman1.jpg}
\caption{Coleman和费曼一样分享“物理学家中的物理学家”这一美誉}
%\label{default}
\end{center}
\end{figure}

他讲授的课程是Physics 253: Quantum Field Theory,即量子场论,量子力学的后续课程之一。在课上,Coleman骄傲地宣称:“Not only God knows, I know, and by the end of the semester, you will know.”

“(关于自然之谜)不仅上帝知道,我知道,而且在本学期末,你们也将知道。”

有讲课视频,有后进整理出来的讲义,这就不再是傲慢,而是乐观、自信,和滔滔不绝,是对人类理性思辨能力的礼赞。

\url{http://arxiv.org/abs/1110.5013}

在另一个场合,Coleman以同样令人欢乐的气氛表达了这种乐观和自信。某次温伯格(Nobel Prize 1979)去哈佛演讲,有人问了个问题,温伯格有些迟疑,说:

“关于这个问题我还不是很确信。”

这时Coleman进来了,他并未听到问题,但听到了温伯格的迟疑,于是他大声喊道:

“我知道答案!”

Coleman边喊边往前走,

“我知道答案。问我吧!问题是什么?”

未闻问题,先摆明答案已经在这里了,这是怎样一种自信啊!!Coleman的潜台词是:问题终将被解决,而且是以物理的方式解决,因此也必是一种简单清晰的解决。

果然,Coleman知道那个问题的答案。

作为普通人学习物理,亲近物理,首先需要欣赏的就是这种乐观、自信的精神气质。不了解这种精英物理学家深入骨髓的精神气质,我们自然无法说自己已经进入物理,已经了解这一理论和精神气质所缔造的现代文明。

这里我要再次重申费曼对基础物理教育的辩护。

费曼曾在Caltech讲授过两年基础物理,但他的尝试并不完全成功,从第二学期起,选课率大幅下降,事实上费曼也知道只有很少的学生怀着极大的兴趣听懂了所有的内容,学得很愉快,而其他学生则可能陷入了麻烦。

尽管有些挫败,费曼仍然在课程结束的时候为自己进行了辩护:

我教这门课的主要目的不是替你为应付考试作准备——甚至也不是替你为日后的职业生涯做准备。我至多希望使你对奇妙的世界以及对物理学家看待这一世界的方式有所了解,我相信这是真正的现代文化的主要部分。或许其他学科的教授会反对这种看法,但我相信他们是完全错误的。

费曼的这一席话很好地概况了物理教育对普通大众的意义,而我也想再次强调存在于物理学中自信、乐观、理性、清晰的精神气质,这是Coleman给我的启迪。

在即将开始我的这个虚拟的量子力学讲座的时候,我愿意以费曼和Coleman为激励,尝试给你,我亲爱的朋友,展示我看待这个奇妙世界的方式,及与之伴随的乐观和自信。

还有就是我滔滔不绝的本领。

~~

(这是我设想的前言,但还是放在后面吧。)

原始网址:\url{http://site.douban.com/223228/widget/notes/14957933/note/329744916/}